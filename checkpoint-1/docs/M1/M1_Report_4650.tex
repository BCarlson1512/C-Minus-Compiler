\documentclass[12pt]{extarticle}
\usepackage[utf8]{inputenc}
\usepackage{cite}
\usepackage{hyperref}
\usepackage[parfill]{parskip}
\usepackage{multicol}
\hypersetup{
    colorlinks=true,
    linkcolor=blue,
    filecolor=magenta,      
    urlcolor=cyan,
    pdftitle={Overleaf Example},
    pdfpagemode=FullScreen,
    }

\title{CIS*4650 Project:\\ Milestone 1}
\author{
    Ben Carlson 1044277 carlsonb@uoguelph.ca \\
    \and 
    Conor Roberts 1056167  crober35@uoguelph.ca \\
}
\date{March 2022}


\begin{document}

\maketitle
\pagebreak
\textbf{Summary (TLDR)}\\

This checkpoint/milestone involved the creation of a scanner, parser and abstract syntax tree for the C- language.
The primary deliverables for this milestone are as follows:

\begin{enumerate}
    \item Create a scanner that generates a sequence of relevant tokens
    \item Create a parser, utilizing CUP that generates and shows an abstract syntax tree
    \item Create robust error recovery, that will report errors and allow the parser to recover
\end{enumerate}

Many principles, including the production rules and scanner regex within cminus.cup and cminus.flex respectively were created from the C- specs file provided.

Our program will take a .cm file, scan it into a relevant stream of tokens via JFlex and then parse it into an AST utilizing CUP.

\textbf{Features of C- Language}
\\ Reserved keywords:
\begin{enumerate}
    \item else, if, while
    \item types: void, int, bool
    \item operators: +, -, /, *
\end{enumerate}
\pagebreak
\textbf{Techniques utilized}
\\
When creating the program, we attempted to follow best practices of OOP, abiding to SOLID. Each stage of the development process was done incrementally, optimistically we planned to test each module separately, however, this changed as the scope of the project became more visible to us. Our development process followed 5 stages with one final debug at the end.

\textbf{Basic Understanding/Design} \\
Following the warm up assignment, we had a working understanding of JFlex and how to utilize it to our advantage. This was only the tip of the iceberg for this milestone, where we had to read docs/experiment with CUP and the provided files to further our understanding. This was done by comparing specs for the tiny language to the c minus specs and then translating rules accordingly. \\
\textbf{Scanner} \\
After getting a base understanding of the file structure and everything related, we started by tackling the scanner portion of the project, considering we already had a solid understanding of JFlex. This required a little bit of preliminary setup in the cminus.cup file so that we could return the correct output tokens as opposed to our previously created token class. \\
\textbf{Absyn} \\
Once the scanner was built, we split into two. The parser and absyn classes could be built without needing each other. Building absyn was primarily done by referencing the lecture slides material. The goal was to build the most in depth, but maintainable class structure since we are only a team of 2 creating this project. \\
\textbf{Parser} \\
We first laid out a skeleton for the CUP rules by translating the specification file provided. Then, we were able to fill in the body of each CUP rule with our class definitions based on our understanding of the content tom be matched. Finally, we added error checking on to as many common error cases as we could so that users would be informed of syntax issues.\\
\textbf{Debugging} \\
We grouped up again and pair programmed the remaining aspects of this milestone as well as debugging the entire program start-to-finish. This program was hard to debug in separate modules, due to the output of Scanner.java and its relation to cup. We ended up clearing all of our errors and warnings before moving onto testing our provided and custom files.

\textbf{Assumptions/Limitations}

This program requires the following resources
\begin{enumerate}
    \item JFlex
    \item Java/JDK
    \item CUP/JCUP
\end{enumerate}

This program will generate a syntax tree of the following code files:
\begin{enumerate}
    \item .cm cminus files
\end{enumerate}

This program can only handle a much smaller subset of keywords within the c programming lanugage. By design to minimize the scope of the project.

\textbf{Improvements}

\begin{enumerate}
    \item Further optimizations within our grammar
    \item More detailed error reporting
\end{enumerate}
\textbf{Contributions}

\begin{multicols}{2}
\textbf{Ben}
\begin{enumerate}
    \item Created AST
    \item Created Tree visitor functions
    \item Wrote Scanner code
    \item Refactored cup rules
    \item created documentation
\end{enumerate}
\columnbreak
\textbf{Conor}
\begin{enumerate}
    \item Implemented CUP rules
    \item Debugged CUP rules
    \item Extended JFlex Rules
    \item Implemented logic and embedded code in CUP
\end{enumerate}
\end{multicols}




\vspace{0.5cm}

\textbf{Acknowledgements} \\
This project was heavily based on/inspired by resources provided by Dr. Fei Song's compilers content. We were provided a rough idea of the implementation process as well as a portion of starter code. The grammars created in CUP are heavily based off the the C- specs. The absyn class structure is heavily based off of lecture content.
\end{document}
